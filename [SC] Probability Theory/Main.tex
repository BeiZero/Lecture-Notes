% !TEX encoding = UTF-8 Unicode
\documentclass[10pt,twoside]{article}
\usepackage[utf8]{inputenc}
\usepackage[left=2cm,right=2cm,top=2cm,bottom=2cm,bindingoffset=0cm]{geometry}
\usepackage[english,russian]{babel}
\usepackage{amsmath,amssymb,amsthm}
\usepackage{hyperref,xcolor}
\hypersetup{pdfstartview=FitH,  linkcolor=linkcolor,urlcolor=urlcolor, colorlinks=true}

\sloppy

\theoremstyle{plain}
\newtheorem{thm}{Теорема}
\newtheorem{lemma}{Лемма}
\newtheorem{corol}{Следствие}
\newtheorem{prop}{Предложение}
\newtheorem{ass}{Утверждение}
\theoremstyle{definition}
\newtheorem{defi}{Определение}
\newtheorem*{remark}{Замечание}
\newtheorem{example}{Пример}

\DeclareMathOperator{\GL}{GL}
\DeclareMathOperator{\cen}{C}
\DeclareMathOperator{\tr}{tr}
\DeclareMathOperator{\Hom}{Hom}
\DeclareMathOperator{\stab}{stab}
\DeclareMathOperator{\Res}{Res}
\DeclareMathOperator{\Ind}{Ind}
\DeclareMathOperator{\Mat}{Mat}
\DeclareMathOperator{\ad}{ad}
\DeclareMathOperator{\Ker}{Ker}
\DeclareMathOperator{\Ima}{Im}
\DeclareMathOperator{\Tr}{Tr}
\DeclareMathOperator{\End}{End}
\DeclareMathOperator{\rank}{rank}
\DeclareMathOperator{\SL}{SL}
\DeclareMathOperator{\Ort}{O}
\DeclareMathOperator{\Aff}{Aff}


\definecolor{linkcolor}{HTML}{0000FF} % цвет ссылок
\definecolor{urlcolor}{HTML}{0000FF} % цвет гиперссылок

\title{Теория Вероятностей}
\author{Автор конспекта \href{http://vk.com/beizero}{Федоров И.И.}\\По лекциям Савинова Е.А.}

\begin{document}
\maketitle
\tableofcontents

\section{Вероятностное пространство}

\begin{defi}
Теория вероятностей -- раздел математики, изучающий случайные события. Теория вероятностей занимается явлениями, которые могут произойти некоторое число раз и обладают свойством статистической устойчивости.
\end{defi}

\begin{defi}
\textit{Случайное событие} -- событие, которое при осуществлении некоторых условий может произойти или не произойти.
\end{defi}

\begin{defi}
Событие называется \textit{достоверным}, если в результате испытаний оно обязательно происходит.
\end{defi}

\begin{defi}
Событие называется \textit{невозможным}, если в результате испытаний оно произойти не может.
\end{defi}

\begin{defi}
События называются \textit{несовместными}, если в результате испытания они не могут произойти вместе.
\end{defi}

\begin{defi}
Совокупность $\mathfrak{A}$ подмножеств множества $\Omega$ называется \textit{алгеброй множеств}, если выполняются условия:
\begin{enumerate}
\item $\Omega\in \mathfrak{A}$
\item $\forall A,B\in \mathfrak{A}\Rightarrow A\cup B\in\mathfrak{A},\bar{A}\in\mathfrak{A}$
\end{enumerate}
\end{defi}

\begin{defi}
Алгебра $\mathfrak{A}$ подмножеств $\Omega$ называется \textit{$\sigma$-алгеброй множеств}, если для любого счетного набора множеств $\{A_i\}_{i\in \mathbb{N}}\in\mathfrak{A}:\bigcup\limits_{i\in \mathbb{N}}A_i\in\mathfrak{A}$.
\end{defi}

\begin{defi}
Пусть $\{\mathfrak{A}_i\}_{i\in I}$ -- набор алгебр подмножеств множества $\Omega$, тогда их пересечение это множество $\{A\subset \Omega:A\in\mathfrak{A}_i,\forall i\in I\}$.
\end{defi}

\begin{ass}
Пересечение $\sigma$-алгебр является $\sigma$-алгеброй.
\end{ass}

\begin{defi}
Пусть $\mathfrak{A}$ -- некоторая совокупность подмножеств $\Omega$, пересечение всех $\sigma$-алгебр содержащих $\mathfrak{A}$ называется наименьшой $\sigma$-алгеброй содержащей $\mathfrak{A}$ или $\sigma$-алгеброй порожденной $\mathfrak{A}$, и обозначается $\sigma(A)$.
\end{defi}

\begin{defi}
Неотрицательная функции $\mu:\mathfrak{A}\to\mathbb{R}_+$ называется \textit{конечно-аддитивной мерой}, если $\forall:A,B\in \mathfrak{A}, A\cap B = \emptyset:\mu(A+B)=\mu(A)+\mu(B)$.
\end{defi}

\begin{defi}
Конечно-аддитивная мера $\mu:\mathfrak{A}\to\mathbb{R}_+$ называется счетно-аддитивной, если $\forall A_1,\dots,A_n,\ldots\in\mathfrak{A},A_i,A_j=\emptyset,i\neq j:\mu\left(\bigcup\limits_{i\in\mathbb{N}}A_i\right)=\sum\limits_{i=1}^\infty\mu(A_i)$.
\end{defi}

\begin{defi}
\textit{Вероятностным пространством} называется тройка $(\Omega,\mathfrak{F},\mathbb{P})$, где $\Omega$ -- пространство элементарных событий, $\mathfrak{F}$ -- $\sigma$-алгебра подмножеств  $\Omega$, 
\end{defi}

\end{document}
