% !TEX encoding = UTF-8 Unicode
\section{Группы Ли}

\begin{defi}
Множество $G$ называется \textit{группой Ли}, если
\begin{enumerate}
\item $G$ — группа,
\item $G$ — гладкое многообразие,
\item отображения $(g,h)\mapsto gh$ и $g\mapsto g^{-1}$ — гладкие.
\end{enumerate}
\end{defi}

\begin{ass}
Пусть группа $G$ — множество решений гладких уравнений, тогда $G$ — группа Ли.
\end{ass}

\begin{examples} Следующие группы являются группами Ли
\begin{enumerate}
\item $\GL_n(\mathbb{R})=\{A\in \Mat_n(\mathbb{R})|\det(A)\neq 0\}$
\item $\SL_n(\mathbb{R})=\{A\in\Mat_n(\mathbb{R})|\det(A) = 1\}$
\item $\Ort_n(\mathbb{R})=\{A\in\Mat_n(\mathbb{R})|A^t=A^{-1}\}$
\item $\Aff(\mathbb{R})=\left\{\left(\begin{array}{cc}a & b \\0 & 1\end{array}\right)\left|\right.a\neq0;a,b\in\mathbb{R}\right\}$
\end{enumerate}
\end{examples}

\begin{defi}
Пусть $G$ — группа Ли, тогда подгруппа $H$ называется \textit{подгруппой Ли}, если $H$ — подмногообразие.
\end{defi}
