% !TEX encoding = UTF-8 Unicode
\section{Гладкие многообразия}

\begin{defi}
Отображение $f:X\to Y$ называется \textit{непрерывным отображением}, если прообраз любого открытого в $Y$ множества открыт в $X$.
\end{defi}

\begin{defi}
Отображение $f:X\to Y$, где $X,Y$ — топологические пространства, называется \textit{гомеоморфизмом}, если:
\begin{enumerate}
  \item $f$ — биекция.
  \item $f$ — непрерывное отображение.
  \item $f^{-1}$ — непрерывное отображение.
\end{enumerate}
Если существует хоть один гомеоморфизм из $X$ в $Y$, то их называют \textit{гомеоморфными}.
\end{defi}

\begin{defi}
Пусть $f:\mathbb{E}^n\to\mathbb{E},a\in\mathbb{E}^n$, причем существует предел
$$\lim_{t\to 0}\frac{f(a+te_i)-f(a)}{t}=f_i'(a)$$
тогда он называется \textit{частной производной}.
\end{defi}

\begin{remark}
Под $\mathbb{E}$ будем подразумевать множество $\mathbb{R}$ или $\mathbb{C}$.
\end{remark}

\begin{defi}
Если частная производная существует в любой точке $a\in\mathbb{E}^n$, то возникает функция ${f_i':\mathbb{E}^n\to\mathbb{E}}$, которая $a$ переводит в $f_i'(a)$.
Функция $f$ называется \textit{гладкой}, если $f'$ - непрерывна.
\end{defi}

\begin{defi}
Отображение ${F:\mathbb{E}^n\to\mathbb{E}^m}$ называется \textit{гладким отображением}, если в координатах оно задается гладкими функциями.
\end{defi}

\begin{ass}
Любое гладкое отображение непрерывно.
\end{ass}

\begin{defi}
Топологическое пространство называется хаусдорфовым, если у любых двух точек существуют непересекающиеся окрестности.
\end{defi}

\begin{defi}
Пусть $M$ — хаусдорфово топологическое пространство со счетной базой, $M$ — представлено в виде объединения своих открытых подмножеств $U_{\alpha}$ и для любого $\alpha$ задан гомеоморфизм $\phi_{\alpha}:U_{\alpha}\to V_{\alpha}$(какое-то открытое подмножество в $\mathbb{E}^n$), причем: $$(\forall \alpha,\beta)(\phi_{\beta}\circ\phi_{\alpha}^{-1}:\phi_{\alpha}(U_{\alpha}\cap U_{\beta})\to\mathbb{E}^n\text{ - гладкое),}$$
тогда $M$ называется \textit{гладким многообразием}. Множества $U_{\alpha}$ называются \textit{картами} на $M$. Отображения ${\phi_{\alpha}:U_{\alpha}\to V_{\alpha}}$ называются \textit{картирующими гомеоморфизмами}. Множество пар $\{(U_\alpha,\phi_\alpha)\}$ называется \textit{атласом} на $M$.
\end{defi}

\begin{defi}
Пусть $M$ — гладкое многообразие, $\{U_\alpha,\phi_\alpha\}$ — атлас на $M$. Рассмотрим конкретное $U_\alpha$ и ${\phi_\alpha:U_\alpha\to V_\alpha\subset\mathbb{E}^n}$. Возникает набор функций ${x^1,\dots, x^n}$, где ${x^i:U_\alpha\to\mathbb{R}:p\mapsto\text{i-ая координата }\phi_\alpha(p)}$. Функции ${\phi_\alpha(p)=(x^1(p),\dots ,x^n(p))}$ называются \textit{локальными координатами} в $U_\alpha$.
\end{defi}

\begin{ass}
Пусть $x^1,\dots, x^n$ — локальные координаты в $U_\alpha$, а $y^1,\dots, y^n$ — локальные координаты в $U_\beta$, тогда в $U_\alpha\cap U_\beta$ можно выразить $y^j$ через $x^i$.\\
Условие гладкости $\phi_\beta\circ\phi_\alpha^{-1}$ означает, что каждое $y^j$ является гладкой функцией от $x^1,\dots, x^n$.
\end{ass}

\begin{thm}
Пусть $X$ — множество решений некоторой системы гладких уравнений
\begin{equation*}
 \begin{cases}
   f_1(x_1,\dots,x_n)=0,
   \\
   \vdots
   \\
   f_m(x_1,\dots,x_n)=0.
 \end{cases}
\end{equation*}
Если ранг матрицы якоби $\left(\frac{\partial f_i}{\partial x_j}\right)$ равен $r$ в каждой точке, то $X$ — гладкое многообразие размерности $n-r$.
\end{thm}

\begin{ass}
Пусть $X$ — гладкое многообразие. Отображение $f:X\to\mathbb{E}$ является \textit{гладким}, если $f\circ \phi_\alpha^{-1}$ гладкое для любого $\alpha$.
\end{ass}

\begin{defi}
Множество всех гладких функций $C^\infty(X)$ с поточечными операциями называется \textit{алгеброй гладких функций}.
\end{defi}

\begin{ass}
Пусть ${X,Y}$ — гладкие мнообразия с атласами  $\{(U_\alpha,\phi_\alpha)\}$  и  $\{(V_\beta,\psi_\beta)\}$. Отображение ${F:X\to Y}$ является гладким, если ${\psi_\beta\circ F\circ\phi_\alpha^{-1}}$ гладкое для любого $\alpha,\beta$.
\end{ass}

\begin{defi}
Пусть ${X,Y}$ — гладкие мнообразия с атласами  $\{(U_\alpha,\phi_\alpha)\}$  и  $\{(V_\beta,\psi_\beta)\}$, тогда произведение $X\times Y$ это гладкое многообразие с атласом  $\{(U_\alpha\times V_\beta,\phi_\alpha\times\psi_\beta)\}$.
\end{defi}

\begin{defi}
Пусть $X$ — гладкое мнообразие, тогда $Y\subset X$ называется \textit{подмногообразием}, если $Y$ является решением системы гладких уравнений и ранг матрицы якоби данной системы равен $r$ в каждой точке.
\end{defi}