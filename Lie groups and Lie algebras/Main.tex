% !TEX encoding = UTF-8 Unicode
\documentclass[10pt,twoside]{article}
\usepackage[utf8]{inputenc}
\usepackage[left=2cm,right=2cm,top=2cm,bottom=2cm,bindingoffset=0cm]{geometry}
\usepackage[english,russian]{babel}
\usepackage{amsmath,amssymb,amsthm}
\usepackage{hyperref,xcolor}
\hypersetup{pdfstartview=FitH,  linkcolor=linkcolor,urlcolor=urlcolor, colorlinks=true}

\usepackage{tikz}
\usetikzlibrary{matrix,arrows,decorations.pathmorphing}

\sloppy

\theoremstyle{plain}
\newtheorem{thm}{Теорема}
\newtheorem{lemma}{Лемма}
\newtheorem{corol}{Следствие}
\newtheorem{prop}{Предложение}
\newtheorem{ass}{Утверждение}
\theoremstyle{definition}
\newtheorem{defi}{Определение}
\newtheorem*{remark}{Замечание}
\newtheorem{example}{Пример}
\newtheorem{examples}{Примеры}


\DeclareMathOperator{\GL}{GL}
\DeclareMathOperator{\cen}{C}
\DeclareMathOperator{\tr}{tr}
\DeclareMathOperator{\Hom}{Hom}
\DeclareMathOperator{\stab}{stab}
\DeclareMathOperator{\Res}{Res}
\DeclareMathOperator{\Ind}{Ind}
\DeclareMathOperator{\Mat}{Mat}
\DeclareMathOperator{\ad}{ad}
\DeclareMathOperator{\Ker}{Ker}
\DeclareMathOperator{\Ima}{Im}
\DeclareMathOperator{\Tr}{Tr}
\DeclareMathOperator{\End}{End}
\DeclareMathOperator{\rank}{rank}
\DeclareMathOperator{\SL}{SL}
\DeclareMathOperator{\Ort}{O}
\DeclareMathOperator{\Aff}{Aff}
\DeclareMathOperator{\Lie}{Lie}


\definecolor{linkcolor}{HTML}{0000FF} % цвет ссылок
\definecolor{urlcolor}{HTML}{0000FF} % цвет гиперссылок

\title{Алгебры и группы Ли}
\author{Автор конспекта \href{http://vk.com/beizero}{Федоров И.И.}\\По лекциям Панова А.Н. и Игнатьева М.В.}

\begin{document}
\maketitle
\tableofcontents

% !TEX encoding = UTF-8 Unicode
\section{Алгебры Ли}

\begin{defi}
\textit{Алгеброй Ли} называется векторное пространство $\mathfrak{g}$ над полем $\mathbb{K}$, снабжённое билинейным отображением $[\cdot,\cdot]:\mathfrak{g}\times\mathfrak{g}\to\mathfrak{g}:(a,b)\mapsto[a,b]$, причем выполняются следующие свойства:
\begin{enumerate}
\item $\forall x \in \mathfrak{g}:[x,x] = 0$;
\item $\forall x,y,z:[x, [y, z]] + [y, [z, x]] + [z, [x, y]] = 0$(тождество Якоби).
\end{enumerate}
Отображение $[\cdot,\cdot]$ обычно называется \textit{коммутатором} или \textit{скобкой Ли}.
\end{defi}

\begin{remark}
В случае если характеристика поля не равна $2$ cвойство $(1)$ можно переписать в виде: $\forall x,y\in\mathfrak{g}:[x,y]=-[y,x]$.
\end{remark}

\begin{remark}
Альтернативная формулировка тождества Якоби: ${[x,[y,z]]=[[x,y],z]+[y,[x,z]]}$.
\end{remark}

\begin{defi}
Пусть $(\mathfrak{g},[\cdot,\cdot])$ — алгебра Ли. Подпространство $\mathfrak{h}\subset\mathfrak{g}$ называется \textit{подалгеброй Ли}, если $(\mathfrak{h},[\cdot,\cdot])$ — алгебра Ли.
\end{defi}

\begin{ass}
Подпространство $\mathfrak{h}\subset\mathfrak{g}$ является подалгеброй Ли $\Leftrightarrow\forall x,y\in\mathfrak{h}:[x,y]\in\mathfrak{h}$.
\end{ass}

\begin{defi}
Множество $\mathfrak{h}$ называется \textit{идеалом}, если $\forall x \in\mathfrak{h},\forall y\in\mathfrak{g}:[x,y]\in\mathfrak{h}$.
\end{defi}

\begin{defi}
Алгебра Ли  $\mathfrak{g}$ называется \textit{простой}, если $\dim\mathfrak{g}>1$ и в ней нет идеалов кроме $\{0\}$ и $\mathfrak{g}$.
\end{defi}

\begin{defi}
Пусть  $\mathfrak{g}_1,\mathfrak{g}_2$ - алгебры Ли. Линейное отображение $\Phi:\mathfrak{g}_1\to\mathfrak{g}_2$ называется \textit{гомоморфизмом}, если $$\forall x,y\in\mathfrak{g}_1:\Phi([x,y])=[\Phi(x),\Phi(y)].$$
Если $\Phi$ - биекция, то $\Phi$ называются \textit{изоморфизмом}.
\end{defi}

\begin{defi}
Пусть $V$ — векторное пространство над $\mathbb{K}$, а $\End(V)$ — пространство эндоморфизмов. Введем на $\End(V)$ скобку Ли формулой $[x,y]=xy-yx$. Полученная алгебра Ли называется \textit{полной линейной алгеброй} и обозначается $\mathfrak{gl}(V)$. Любая подалгебра в $\mathfrak{gl}(V)$ называется \textit{линейной алгеброй Ли}.
\end{defi}

\begin{defi}
Пусть $V$ — векторное пространство над полем $\mathbb{K}$. \textit{Представлением алгебры Ли} $\mathfrak{g}$ называется отображение ${\tau:\mathfrak{g}\to\mathfrak{gl}(V):x\mapsto\tau_x}$ такое, что
\begin{enumerate}
\item $\tau_{x+y}=\tau_x+\tau_y$,
\item $\tau_{\alpha x} = \alpha\tau_x$,
\item $\tau_{[x,y]}=[\tau_x,\tau_y]$.
\end{enumerate}
\end{defi}

\begin{remark}
Представление алгебры Ли $\mathfrak{g}$ это просто гомоморфизм алгебр Ли $\tau:\mathfrak{g}\to\mathfrak{gl}(V)$.
\end{remark}

\begin{ass}
Пусть $\mathfrak{g}=\mathfrak{gl}(n,\mathbb{K})$, а $V=\Mat(n,\mathbb{K})$, тогда $\tau_X(A)=XA+AX^T$ - представление.
\end{ass}

\begin{defi}
Пусть $\mathfrak{g}$ — алгебра Ли, а $V=\mathfrak{g}$, тогда гомоморфизм $\ad:\mathfrak{g}\to\mathfrak{gl}(V):x\mapsto \ad_x$ такой, что $\ad_x(y) = [x,y]$, называется \textit{присоединенным представлением}.
\end{defi}

\begin{defi}
Пусть $\mathfrak{g}$ — алгебра Ли, а $I\subset\mathfrak{g}$ — идеал. Фактормножество $\mathfrak{g}$ по отношению эквивалентности "$x\equiv y\mod I \Leftrightarrow x-y\in I$" $ $ называется \textit{факторалгеброй Ли} и обозначается $\mathfrak{g}/I$. Класс эквивалентности элемента $x$ обычно обозначается $x+I$ или $[x]$.
\end{defi}

\begin{ass}
Факторалгебра Ли сама является алгеброй Ли: $[x+I,y+I]=[x,y]+I$.
\end{ass}

\begin{defi}
Гомоморфизм $\pi:\mathfrak{g}\to\mathfrak{g}/I:x\mapsto x+I$ называется \textit{канонической(естественной) проекцией}.
\end{defi}

\begin{defi}
Пусть $\Phi:\mathfrak{g}_1\to\mathfrak{g}_2$ - гомоморфизм, тогда

$\Ker\Phi:=\{x\in\mathfrak{g}_1|\Phi(x)=0\}$ - ядро $\Phi$,

$\Ima\Phi:=\{y\in\mathfrak{g}_2|\exists x\in\mathfrak{g}_1:\Phi(x)=y\}$ - образ $\Phi$.
\end{defi}

\begin{ass}
Множество $\Ker\Phi$ — идеал в $\mathfrak{g}_1$, а $\Ima\Phi$ - подалгебра в $\mathfrak{g}_2$.
\end{ass}

\begin{thm}
Пусть ${\Phi:\mathfrak{g}_1\to\mathfrak{g}_2}$ — гомоморфизм, тогда существует изоморфизм ${\psi:\mathfrak{g}_1/\Ker\Phi\to\Ima\Phi}$.
\end{thm}

\begin{defi}
Пусть $\mathfrak{g}$ — алгебра Ли, тогда $\beta:\mathfrak{g}\times\mathfrak{g}\to\mathbb{K}:(x,y)\mapsto\Tr(\ad_x\ad_y)$ называется \textit{формой Киллинга}.
\end{defi}

\begin{remark}
На английском языке ''форма Киллинга'' записывается как ''Killing form'', что дословно переводится как ''убивающая форма''.
\end{remark}

\begin{ass}
Отображение $\beta$ — симметрическая билиненая форма, причем $(x,[y,z])=([x,y],z)$.
\end{ass}

% !TEX encoding = UTF-8 Unicode
\section{Гладкие многообразия}

\begin{defi}
Отображение $f:X\to Y$ называется \textit{непрерывным отображением}, если прообраз любого открытого в $Y$ множества открыт в $X$.
\end{defi}

\begin{defi}
Отображение $f:X\to Y$, где $X,Y$ — топологические пространства, называется \textit{гомеоморфизмом}, если:
\begin{enumerate}
  \item $f$ — биекция.
  \item $f$ — непрерывное отображение.
  \item $f^{-1}$ — непрерывное отображение.
\end{enumerate}
Если существует хоть один гомеоморфизм из $X$ в $Y$, то их называют \textit{гомеоморфными}.
\end{defi}

\begin{defi}
Пусть $f:\mathbb{E}^n\to\mathbb{E},a\in\mathbb{E}^n$, причем существует предел
$$\lim_{t\to 0}\frac{f(a+te_i)-f(a)}{t}=f_i'(a)$$
тогда он называется \textit{частной производной}.
\end{defi}

\begin{remark}
Под $\mathbb{E}$ будем подразумевать множество $\mathbb{R}$ или $\mathbb{C}$.
\end{remark}

\begin{defi}
Если частная производная существует в любой точке $a\in\mathbb{E}^n$, то возникает функция ${f_i':\mathbb{E}^n\to\mathbb{E}}$, которая $a$ переводит в $f_i'(a)$.
Функция $f$ называется \textit{гладкой}, если $f'$ - непрерывна.
\end{defi}

\begin{defi}
Отображение ${F:\mathbb{E}^n\to\mathbb{E}^m}$ называется \textit{гладким отображением}, если в координатах оно задается гладкими функциями.
\end{defi}

\begin{ass}
Любое гладкое отображение непрерывно.
\end{ass}

\begin{defi}
Топологическое пространство называется хаусдорфовым, если у любых двух точек существуют непересекающиеся окрестности.
\end{defi}

\begin{defi}
Пусть $M$ — хаусдорфово топологическое пространство со счетной базой, $M$ — представлено в виде объединения своих открытых подмножеств $U_{\alpha}$ и для любого $\alpha$ задан гомеоморфизм $\phi_{\alpha}:U_{\alpha}\to V_{\alpha}$(какое-то открытое подмножество в $\mathbb{E}^n$), причем: $$(\forall \alpha,\beta)(\phi_{\beta}\circ\phi_{\alpha}^{-1}:\phi_{\alpha}(U_{\alpha}\cap U_{\beta})\to\mathbb{E}^n\text{ - гладкое),}$$
тогда $M$ называется \textit{гладким многообразием}. Множества $U_{\alpha}$ называются \textit{картами} на $M$. Отображения ${\phi_{\alpha}:U_{\alpha}\to V_{\alpha}}$ называются \textit{картирующими гомеоморфизмами}. Множество пар $\{(U_\alpha,\phi_\alpha)\}$ называется \textit{атласом} на $M$.
\end{defi}

\begin{defi}
Пусть $M$ — гладкое многообразие, $\{U_\alpha,\phi_\alpha\}$ — атлас на $M$. Рассмотрим конкретное $U_\alpha$ и ${\phi_\alpha:U_\alpha\to V_\alpha\subset\mathbb{E}^n}$. Возникает набор функций ${x^1,\dots, x^n}$, где ${x^i:U_\alpha\to\mathbb{R}:p\mapsto\text{i-ая координата }\phi_\alpha(p)}$. Функции ${\phi_\alpha(p)=(x^1(p),\dots ,x^n(p))}$ называются \textit{локальными координатами} в $U_\alpha$.
\end{defi}

\begin{ass}
Пусть $x^1,\dots, x^n$ — локальные координаты в $U_\alpha$, а $y^1,\dots, y^n$ — локальные координаты в $U_\beta$, тогда в $U_\alpha\cap U_\beta$ можно выразить $y^j$ через $x^i$.\\
Условие гладкости $\phi_\beta\circ\phi_\alpha^{-1}$ означает, что каждое $y^j$ является гладкой функцией от $x^1,\dots, x^n$.
\end{ass}

\begin{thm}
Пусть $X$ — множество решений некоторой системы гладких уравнений
\begin{equation*}
 \begin{cases}
   f_1(x_1,\dots,x_n)=0,
   \\
   \vdots
   \\
   f_m(x_1,\dots,x_n)=0.
 \end{cases}
\end{equation*}
Если ранг матрицы якоби $\left(\frac{\partial f_i}{\partial x_j}\right)$ равен $r$ в каждой точке, то $X$ — гладкое многообразие размерности $n-r$.
\end{thm}

\begin{ass}
Пусть $X$ — гладкое многообразие. Отображение $f:X\to\mathbb{E}$ является \textit{гладким}, если $f\circ \phi_\alpha^{-1}$ гладкое для любого $\alpha$.
\end{ass}

\begin{defi}
Множество всех гладких функций $C^\infty(X)$ с поточечными операциями называется \textit{алгеброй гладких функций}.
\end{defi}

\begin{ass}
Пусть ${X,Y}$ — гладкие мнообразия с атласами  $\{(U_\alpha,\phi_\alpha)\}$  и  $\{(V_\beta,\psi_\beta)\}$. Отображение ${F:X\to Y}$ является гладким, если ${\psi_\beta\circ F\circ\phi_\alpha^{-1}}$ гладкое для любого $\alpha,\beta$.
\end{ass}

\begin{defi}
Пусть ${X,Y}$ — гладкие мнообразия с атласами  $\{(U_\alpha,\phi_\alpha)\}$  и  $\{(V_\beta,\psi_\beta)\}$, тогда произведение $X\times Y$ это гладкое многообразие с атласом  $\{(U_\alpha\times V_\beta,\phi_\alpha\times\psi_\beta)\}$.
\end{defi}

\begin{defi}
Пусть $X$ — гладкое мнообразие, тогда $Y\subset X$ называется \textit{подмногообразием}, если $Y$ является решением системы гладких уравнений и ранг матрицы якоби данной системы равен $r$ в каждой точке.
\end{defi}
% !TEX encoding = UTF-8 Unicode
\section{Группы Ли}

\begin{defi}
Множество $G$ называется \textit{группой Ли}, если
\begin{enumerate}
\item $G$ — группа,
\item $G$ — гладкое многообразие,
\item отображения $(g,h)\mapsto gh$ и $g\mapsto g^{-1}$ — гладкие.
\end{enumerate}
\end{defi}

\begin{ass}
Пусть группа $G$ — множество решений гладких уравнений, тогда $G$ — группа Ли.
\end{ass}

\begin{examples} Следующие группы являются группами Ли
\begin{enumerate}
\item $\GL_n(\mathbb{R})=\{A\in \Mat_n(\mathbb{R})|\det(A)\neq 0\}$
\item $\SL_n(\mathbb{R})=\{A\in\Mat_n(\mathbb{R})|\det(A) = 1\}$
\item $\Ort_n(\mathbb{R})=\{A\in\Mat_n(\mathbb{R})|A^t=A^{-1}\}$
\item $\Aff(\mathbb{R})=\left\{\left(\begin{array}{cc}a & b \\0 & 1\end{array}\right)\left|\right.a\neq0;a,b\in\mathbb{R}\right\}$
\end{enumerate}
\end{examples}

\begin{defi}
Пусть $G$ — группа Ли, тогда подгруппа $H$ называется \textit{подгруппой Ли}, если $H$ — подмногообразие.
\end{defi}




\end{document}